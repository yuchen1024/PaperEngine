\ifWidePage
  \documentclass[aspectratio=169, xcolor=table]{beamer}
\else
  \documentclass[xcolor=table]{beamer}
\fi


\usepackage{fontspec}
\usepackage{tcolorbox}

\usepackage{xeCJK}
%\setCJKmainfont{KaiTi_GB2312}
\setCJKmainfont{Kai}
% \setCJKmainfont{Kai}[AutoFakeBold]
%\usefonttheme{serif}
\usefonttheme{professionalfonts}

\usetheme{default} % 加载默认样式文件
%\usefonttheme{serif} % 使用衬线字体
\usecolortheme{orchid} % 兰花色彩风格
\useinnertheme{rounded} % 圆角风格


\setbeamerfont*{frametitle}{size=\normalsize,series=\bfseries} % 设定页眉标题
\setbeamertemplate{navigation symbols}{} % 设定导航标志为空
% \setbeamertemplate{headline}[default]
%\setbeamertemplate{navigation symbols}{\Large{\copyright Yu Chen}} % 设定导航标志
\setbeamertemplate{footline}[frame number] % 设定页脚

% \setbeamersize
% {
%   text margin left=1cm,
%   text margin right=1cm
% }

% 加载各类宏包
\usepackage{amsmath, amssymb, amsfonts} % 数学类AMS宏包
\usepackage{mathrsfs} % 数学字体
\usepackage{tabularx} % 表格
\usepackage{threeparttable} % 三线表
\usepackage{array}
\usepackage{multirow}
\usepackage{makecell} % 处理表格中分行分栏
\usepackage{diagbox}
\usepackage{soul}
\usepackage{cancel,siunitx}
\usepackage{multimedia}
\usepackage{caption}


\usepackage[ruled,lined,linesnumbered]{algorithm2e} % 算法框图
% \let\oldnl\nl% Store \nl in \oldnl
% \newcommand{\nonl}{\renewcommand{\nl}{\let\nl\oldnl}}% Remove line number for one line
\IncMargin{1em}
\SetNlSty{text}{}{:}

\usepackage{wasysym} % 表情图标
\usepackage{pifont}  % 特殊符号图标(如对勾)
\usepackage{animate} % 动图

\usepackage{framed} % 加框包
\usepackage{boxedminipage}

%\usepackage{graphicx} % 插图包
\usepackage{xcolor}   % 增强色彩包
\usepackage{tikz}     % 绘图包
\usetikzlibrary{calc, matrix, patterns, backgrounds, arrows, shapes, snakes, circuits, circuits.logic.US, 
chains, fit, decorations, intersections, positioning}

\tikzset{
    shapenode/.style = {draw, rectangle, fill=none, auto, node distance=0em, font=\normalsize}, 
    roundnode/.style = {draw, rectangle, rounded corners=0.5em, inner sep = 0.2em, 
      node distance=0em, minimum height=1.5em},
    rectanglenode/.style = {draw, rectangle, inner sep = 0em, node distance=0em, minimum height=2em},
    textnode/.style  = {draw=none, fill=none, rectangle, minimum size=0cm, inner sep = 0em, 
      auto, node distance=0em, font=\normalsize},
    smalltextnode/.style  = {draw=none, fill=none, rectangle, minimum size=0.5cm, auto, node distance=0em, font=\small},
    circlenode/.style = {circle, draw, auto, inner sep = 0em, node distance=0em}, 
    ellipsenode/.style = {draw, ellipse, inner sep = 0em, node distance=0em, minimum height=0em, minimum width=0em},
    dotnode/.style = {draw, fill=black, circle, auto, minimum size=0.2em, inner sep = 0em, node distance=0em},  
    connect/.style = {black,->},
    implies/.style={double,double equal sign distance,-implies}, 
    every text node part/.style={align=center}
}

\usepackage{url} % 设置参考文献样式
%\hypersetup{colorlinks=true, anchorcolor=blue, linkcolor=blue, citecolor=blue, urlcolor=cyan}

\usepackage[english]{babel}


\let\oldcite=\cite                                                              
\renewcommand{\cite}[1]{\blue{\oldcite{#1}}}

\renewcommand\bibname{References}
% \renewcommand{\bibfont}{\scriptsize}


\newcommand{\cmark}{\ding{51}}%
\newcommand{\xmark}{\ding{55}}%

\newcommand{\dash}{\rule{0.05em}{0pt}\rule[0.3em]{0.6em}{0.06em}\rule{0.05em}{0pt}}
\newcommand{\define}{\stackrel{\text{def}}{=}}


% \usepackage[printwatermark]{xwatermark} % 嵌入水印
% \newsavebox\mybox
% \savebox\mybox{\tikz[color=blue!30,opacity=0.15]\node{Yu Chen};}
% \newwatermark*[
%   allpages,
%   angle=40,
%   scale=9,
%   xpos=-24,
%   ypos=32
% ]{\usebox\mybox} % 水印设置

\definecolor{lightmidnightblue}{rgb}{0.2, 0.2, 0.7}
\definecolor{mycolor}{rgb}{51,51,178}
\definecolor{darkgreen}{rgb}{0.0,0.5,0.0}

\newcommand{\blue}[1]{{\color{blue} #1}}
\newcommand{\white}[1]{{\color{white} #1}}
\newcommand{\red}[1]{{\color{red} #1}}
\newcommand{\cyan}[1]{{\color{cyan} #1}}
\newcommand{\green}[1]{{\color{green}#1}}
\newcommand{\darkgreen}[1]{{\color{darkgreen}#1}}
\newcommand{\orange}[1]{{\color{orange} #1}}
\newcommand{\yellow}[1]{{\color{yellow} #1}}
\newcommand{\magenta}[1]{{\color{magenta} #1}}
\newcommand{\gray}[1]{{\color{gray} #1}}
\newcommand{\purple}[1]{{\color{purple} #1}}
\newcommand{\mdblue}[1]{{\color{lightmidnightblue} #1}}

\newcommand{\codecomment}[1]{{\color{blue} //#1}}

\newcommand{\splitline}{\darkgreen{\rule[0.25\baselineskip]{\textwidth}{0.6pt}}} % 分割线

\usepackage[normalem]{ulem} % 批注

% \newcommand{\redboxed}[1]{\red{\boxed{#1}}}
% \newcommand{\blueboxed}[1]{\mdblue{\boxed{#1}}}
\newcommand{\redul}{\bgroup\markoverwith{\textcolor{red}{\rule[-0.5ex]{2pt}{0.4pt}}}\ULon} % 红色下划线
\newcommand{\blueul}{\bgroup\markoverwith{\textcolor{blue}{\rule[-0.5ex]{2pt}{0.4pt}}}\ULon} % 蓝色下划线
\newcommand{\greenul}{\bgroup\markoverwith{\textcolor{green}{\rule[-0.5ex]{2pt}{0.4pt}}}\ULon} % 绿色下划线
\newcommand{\darkgreenul}{\bgroup\markoverwith{\textcolor{darkgreen}{\rule[-0.5ex]{2pt}{0.4pt}}}\ULon} % 绿色下划线
\newcommand{\blackul}{\bgroup\markoverwith{\textcolor{black}{\rule[-0.5ex]{2pt}{0.4pt}}}\ULon} % 黑色下划线


\makeatletter % 批注颜色
\newcommand\soulcolor{%
  \let\set@color\beamerorig@set@color
  \let\reset@color\beamerorig@reset@color}
\makeatother

\setbeamerfont{footnote}{size=\small}
% add text to beamer footline
\makeatletter
% add a macro that saves its argument
\newcommand{\footlineextra}[1]{\gdef\insertfootlineextra{#1}}
\newbox\footlineextrabox
 
% add a beamer template that sets the saved argument in a box.
% The * means that the beamer font and color "footline extra" are automatically added. 
\defbeamertemplate*{footline extra}{default}{
  \begin{beamercolorbox}[ht=2.25ex,dp=1ex,leftskip=\Gm@lmargin]{footline extra}
    \insertfootlineextra
    % \par\vspace{2.5pt}
  \end{beamercolorbox}
}
 
\addtobeamertemplate{footline}{%
  % set the box with the extra footline material but make it add no vertical space
  \setbox\footlineextrabox=\vbox{\usebeamertemplate*{footline extra}}
  \vskip -\ht\footlineextrabox
  \vskip -\dp\footlineextrabox
  \box\footlineextrabox%
}
{}
 
% patch \begin{frame} to reset the footline extra material
\let\beamer@original@frame=\frame
\def\frame{\gdef\insertfootlineextra{}\beamer@original@frame}
\footlineextra{}
\makeatother

\newcommand{\iO}{i\mathcal{O}}
\newcommand{\diO}{di\mathcal{O}}
\newcommand{\sample}{\xleftarrow{\textup{\tiny R}}}
\newcommand{\AdvA}{\mathsf{Adv}_\mathcal{A}(\lambda)}

\newcommand{\expect}{\mathbb{E}}
\newcommand{\entropy}{\mathsf{H}}
\newcommand{\minentropy}{\mathsf{H}_\infty}
\newcommand{\avminentropy}{\tilde{\mathsf{H}}_\infty}
\newcommand{\Phisb}{\Phi_\textup{brs}^\textup{srs}}

\newcommand{\Osign}{\mathcal{O}_\mathsf{sign}}
\newcommand{\Odec}{\mathcal{O}_\mathsf{dec}}

\setbeamertemplate{theorems}[numbered] %强制定理编号

% define theorem enviroments
\theoremstyle{plain}% default
% \newtheorem{claim}[theorem]{Claim}
\newtheorem{proposition}[theorem]{Proposition}

% \theoremstyle{definition}
% \newtheorem{definition}{Definition}[section]
% \newtheorem{conjecture}{Conjecture}[section]
% \newtheorem{example}{Example}[section]
% \newtheorem{note}{Note}[section]
% \newtheorem{assumption}{Assumption}[section]
% \newtheorem{construction}{Construction}[section]
% \newtheorem{exercise}{Exercise}[section]
% \newtheorem{discussion}{Discussion}[section]



\newenvironment{remarkblock}[0]{%
\setbeamercolor{block body}{bg=blue!10,fg=black}
\begin{block}{}}
{\end{block}}

\newenvironment{noteblock}[0]{%
\setbeamercolor{block body}{bg=gray!20,fg=black}
\begin{block}{}}
{\end{block}}


\newenvironment{claim}
{\begin{trivlist} \item \textbf{Claim.}}
{\end{trivlist}}

% 定义假设环境
\newenvironment<>{assumption}[1]{%
  \setbeamercolor{block title}{bg=orange,fg=white}
  \setbeamercolor{block body}{bg=orange!20,fg=black}
  \begin{block}#2{#1}}{\end{block}
}

% 定义构造环境
\newenvironment<>{myconstruction}[1]{%
  \setbeamercolor{block title}{bg=orange,fg=white}
  \setbeamercolor{block body}{bg=orange!20,fg=black}
  \begin{block}#2{#1}}{\end{block}
}

\makeatletter
\def\th@mystyle{%
    \normalfont % body font
    \setbeamercolor{block title example}{}
    \setbeamercolor{block body example}{}
    \def\inserttheoremblockenv{exampleblock}
  }
\makeatother
\theoremstyle{mystyle}
\newtheorem*{construction}{Construction}

\newenvironment<>{greenbox}[1]{
\begin{tcolorbox}[colback=green!5,colframe=green!40!black]#1}
{\end{tcolorbox}}

\newenvironment<>{bluebox}[1]{
\begin{tcolorbox}[boxsep=0em,colback=white,colframe=blue]#1}
{\end{tcolorbox}}

\newenvironment<>{redbox}[1]{
\begin{tcolorbox}[boxsep=0em, colback=white,colframe=red]#1}
{\end{tcolorbox}}

\newenvironment<>{titledredbox}[1][2]{
\begin{tcolorbox}[boxsep=0em, colback=white,colframe=red, title=#1]#2}
{\end{tcolorbox}}

\newenvironment<>{orangebox}[1]{
\begin{tcolorbox}[colback=white,colframe=orange]#1}
{\end{tcolorbox}}

\newenvironment<>{question}[1]{
\begin{center}\itshape{#1}}
{\end{center}}

\newenvironment<>{theoremblock}[1]{
\begin{noteblock}\mdblue{Theorem:} #1}
{\end{noteblock}}

\newenvironment<>{lemmablock}[1]{
\begin{noteblock}\mdblue{Lemma:} #1}
{\end{noteblock}}

\newcommand{\Cloud}[2][180]% [angle], content
{\begin{tikzpicture}[overlay]
    \node[align=center, draw, shading=ball, text=white, cloud callout, cloud puffs=17, cloud puff arc=140, 
      callout pointer segments=3, anchor=pointer, callout relative pointer={(#1:2 cm )}, aspect=4,scale=0.5] at (0.2ex,0.5ex) {#2};
\end{tikzpicture}
}

\newcommand{\mydash}{\rule{0.05em}{0pt}\rule[0.35em]{0.5em}{0.05em}}
